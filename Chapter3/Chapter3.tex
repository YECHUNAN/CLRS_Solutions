\documentclass[]{article}
\usepackage[linesnumbered,boxed]{algorithm2e}
\usepackage{amsmath}
\usepackage{graphicx}
\usepackage{multirow}
\usepackage{array}
%opening
\title{}
\author{}

\begin{document}
\title{Chapter 3 solution}
\maketitle
\begin{enumerate}
\item[3.1-1] Since f(n) and g(n) are both nonnegative\\
$ \exists c1=1>0, k1>0, \forall n>k1, \max(f(n),g(n)<=f(n)+g(n) $\\
$ \exists c2=0.5>0, k2>0, \forall n>k2, \max(f(n),g(n)>=0.5*(f(n)+g(n)) $\\
Therefore, $ \max(f(n),g(n)) = \Theta(f(n)+g(n)) $

\item[3.1-2] $ \frac{(n+a)^b}{n^b} = (1+\frac{a}{n})^b \in (1/2, 2)\ \  \text{as}\ \  n \to \infty$

Thus $(n+a)^b = \Theta(n^b)$

\item[3.1-3] It is meaningless because the $O$ notation is a upper bound, while "At least" is supposed to demonstrate some lower bound of the running time.

\item[3.1-4] $2^{n+1} = O(2^n)$ since $2^{n+1} < 3*2^{n}$\\
$2^{2n} \neq O(2^n)$ since $\frac{2^{2n}}{2^n} = 2^n \ \ \to \ \ \infty$ as n goes to infinity

\item[3.1-5] Prove that $f(n) = \Theta(g(n))$ if and only if\\ $f(n) = O(g(n))$ and $ f(n) = \Omega(g(n)) $

Just rewrite the definitions and it is easy to find how this relationship holds.

\item[3.1-6] If and algorithm has running time $\Theta(g(n))$, then it must be both $O(g(n))$ and $\Omega(g(n))$\\
Since it includes the worst case and the best case, then it is trivial that $ T_{worst} = O(g(n)) $ and $T_{best} = \Omega(g(n))$

\item[3.1-7] Prove that $o(g(n) \cap \omega(g(n)) = \emptyset $\\
Suppose there is a function $f(n)$ in this intersection set. Then by the property of these two notations, we can see that $$ \lim\limits_{n\to \infty} \frac{f(n)}{g(n)} = 0\ \  \wedge  \ \ \lim\limits_{n\to \infty}\frac{f(n)}{g(n)} > 0 $$
which are mutually exclusive. 

\item[3.1-8] Definition of $\Omega(g(n,m))$\\
$ \Omega(g(n,m)) = \\ \{ f(n,m): \exists \text{c,n0,m0$>$0 s.t.} f(n,m)>=cg(n,m) \forall n>=n0 \wedge m>=m0 \}$

$\Theta(g(n,m)) = \Omega(g(n,m)) \cap O(g(n,m)) $

\item[3.2-1] Proof: \\
By given condition, we know that\\ $\forall x>y,\ \ f(x)>f(y)$ and $\forall x>y,\ \ f(x)>f(y)$\\
similar for g(n)

Then it is easy to see that\\ $ \forall x>y, f(x)+g(x)>f(y)+g(y) $\\
$ \forall x>y,\ \ g(x)>g(y)\implies f(g(x))>f(g(y)) $\\
If f and g are both nonnegative, then their product will be trivially increasing.

\item[3.2-2] Prove that $ a^{\log_b(c)} = c^{\log_b(a)} $\\
Suppose the equation holds. Take log a on both sides of the equation, we get $$ \log_b(c) = \log_b(a) \log_a(c) $$

Decompose the log terms with the relation that $ \log_x(y) = \frac{\lg x}{\lg y} $, then it is easy to see that the left hand side and the right side are equivalent.

Since the process above is revertible, it can be deduced that the original equation holds.

\item[3.2-3] Prove that $\lg(n!) = \Theta(n\lg n)$\\
Proof: $\frac{\lg(n!)}{n\lg(n)} = \frac{\lg(n!)}{\lg(n^n)}$\\

It is easy to say that this term is greater than 0 and is upper-bounded by 1. So the O relation is obvious, now we need to show that this term doesn't converge to 0. 

By Stirling Approximation, 
$$
n! = \sqrt{2\pi n}(\frac{ n}{e})^{n}\Theta(1)
$$

$$ \lg(n!) = 1/2*\lg n + \Theta(\lg(n^n)) = \Theta(\lg(n^n))$$


The sum excluding $f(k)$ in the first equation gives $ \Theta(\lg m) $ which is equivalent to $ \Theta(n) $ 



Then it is trivial to see that $ \lg(n!) = \Theta(n\lg n) $

Prove: $ n! = \omega(2^n) $\\
Proof: it is true that $ 4! > 2^4 $
Starting from $n=4$, suppose that $ \frac{n!}{2^n} $ converges to some constant c (say equals $ c-e $ for some infinitesimally small $ e $) when n is greater than k. We can always find some n such that$(n+1)/2 > \frac{c+1}{c-e}$ such that $\frac{(n+1)!}{2^{n+1}} > c$ which contradicts the condition for converging. We can deduce that there can't be upper bound for the result of this division and then we can state that 
$$
n! = \omega(2^n)
$$

Prove: $ n! = o(n^n) $\\
This can be proved by deducing that $\frac{n!}{n^n}$ is monotonically decreasing and it is bounded by 0 and 1.

\item[3.2-4] Is the function $ \lceil \lg(n) \rceil ! $ polynomially bounded?\\
No this is not polynomially bounded. Recall the result we have proved in 3.2-3, $$ \lg( \lceil \lg(n) \rceil ! ) = \omega( 2^{\lceil \lg(n) \rceil} ) = \omega (n) $$

However, for any polynomial P(n), $$ \lg(P(n)) = \Theta(n) $$

Thus it is obvious that this function is asymptotically larger than any polynomial.

Is the function $ \lceil \lg \lg(n) \rceil ! $ polynomially bounded?\\

Yes, this function is polynomially bounded. By Stirling's approximation, 
$$
\lceil\lg\lg n\rceil! = t! = \sqrt{2\pi t}(\frac{t}{e})^{t}(1+\Theta(1/t))
$$

$$ \lg (t!) = O(t \lg (t) ) = O(\lceil\lg\lg n\rceil \lceil\lg\lg\lg n\rceil) = O(2^{\lceil\lg\lg n\rceil}) = O(\lg n)$$

while $$ \lg P(n) = \Theta(n)  $$

Thus it is polynomially bounded.

\item[3.2-5] Which is asymptotically larger? $ \lg(\lg^* n) $ or $ \lg^*(\lg n) $

Actually, $ \lg^*(\lg n) = \lg^* n -1 = \Theta(\lg^* n)$

But $ \lg(\lg^* n) = \Theta(\lg(\lg^* n)) $

Thus the second function is asymptotically larger.

\item[3.2-6] Since we already know the value of $\phi$ and $\hat{\phi}$, just plug them into the equation and we can see the satisfaction.

\item[3.2-7] Show that $ F_i = \frac{\phi^i-\hat{\phi^i}}{\sqrt{5}} $ by induction

When $i=0$, $F_0 = 0$\\
Suppose $ F_k =  \frac{\phi^k-\hat{\phi^k}}{\sqrt{5}} $ and $ F_{k-1} =  \frac{\phi^{k-1}-\hat{\phi^{k-1}}}{\sqrt{5}} $, \\then 
$F_{k+1} = F_{k-1} + F_k = \frac{\phi^{k-1}-\hat{\phi^{k-1}} + \phi^k-\hat{\phi^k}}{\sqrt{5}} $

extract the reciprocal of the k-1 term and add it with k term, it is easy to see that this is exactly the form of k+1. Then by induction, we show that this is the formula for Fibonacci number for all i.

\item[3.2-8] Show that $ k\ln(k) = \Theta(n) $ implies that $ k=\Theta(n/\ln(n)) $

Take ln on both side of the given relation
$ \ln k + \ln\ln k = \Theta(\ln n) $\\
The left hand side has $ \ln k+\ln\ln k = \Theta(\ln n) $\\
Thus $ \Theta(\ln k) = \Theta(\ln n) \implies \ln k= \Theta (\ln n) $\\
This works because this is a two sided bound instead of O or $\Omega$ and taking lg will shrink the range of both sides of the equation instead of expanding them.

By taking the reciprocal of the two sided constants, we can see that\\
 $ \frac{1}{\ln k} = \Theta(\frac{1}{\ln n}) $

Multiply with the original relation, we get\\
$ k = \Theta(n/ \ln n) $

\item[3-1] $ p(n) = \sum\limits_{i=0}^{d} a_in^i ,\ \ a_d>0$\\
Prove the following statements:\\
a. If $k>=d$ then $ p(n) = O(n^k) $\\
Since $n^k$ is asymptotically larger than the sum of $ 1+n+n^2+...+n^{k-1} >= 1+n+n^2+...+n^{d} $\\
Let the constant $ c=\sum\limits_{i=0}^{d} |a_i| $ then write the definition of the O relation. It will be easy to see that $ p(n) < c*n^k $. Thus $ p(n) = O(n^k) $

b. $\frac{n^k}{p(n)} = 1 + c_1/n+... <2 $ as n goes to infinity since $k<=d$. Reverse the inequality and it's easy to show that $p(n)=\Omega(n^k)$

c. Combine the result of a and b will be sufficient.

d. This directly follows the method of b. Just show that $ \frac{p(n)}{n^k} \to 0 $ as n goes to infinity.

e. Similar to d but in the converse way.

\item[3-2] Relative asymptotic growths

Whether A is *(B) or not.\\
\begin{tabular}{|c|c|c|c|c|c|c|}
\hline
A & B & O & o & $ \Omega $ & $\omega$ & $ \Theta $\\
\hline
$\lg^k n$ & $n^{\epsilon} $ & yes & yes & no & no & no\\
\hline
$ n^k $ & $c^n$ & yes & yes & no & no & no\\
\hline
$\sqrt{n}$ & $ n^{\sin(n)} $ & no & no & no & no & no\\
\hline
$2^n$ & $ 2^{n/2} $ & no & no & yes & yes & no\\
\hline
$ n^{\lg c} $ & $ c^{\lg n} $ & yes & no & yes & no & yes\\
\hline
$ \lg(n!) $ & $\lg(n^n)$ & yes & no & yes & no & yes\\
\hline
\end{tabular}

\item[3-3] I get a sorted list from CLRS study group on its website, I checked some pairs but didn't covered all the checking tasks. Some comparisons are still confusing to me.

From asymptotically smallest to largest:\\
1. $ 1 = n^{1/\lg n} $\\
2. $ \lg(\lg^* n) $\\
3. $ \lg^*(n) = \lg^*(\lg n) $\\
4. $ 2^{\lg^* n} $\\
5. $ \ln \ln n $\\
6. $ \sqrt{\lg n} $\\
7. $ \ln n $\\
8. $ \lg^2(n) $\\
9. $ 2^{\sqrt{2\ln n}} $\\
10.$ (\sqrt{n})^{\lg n} $\\
11.$ n=2^{\lg n} $\\
12.$ \lg(n!) = n\lg n$\\
13.$ n^2 = 4^{\lg n}  $\\
14.$ n^3 $\\
15.$ n^{\lg \lg n} = (\lg n)^{\lg n} $\\
16.$ (\frac{3}{2})^{n} $\\
17.$ n2^n $\\
18.$ e^n $\\
19.$ n! $\\
20.$ (n+1)! $\\
21.$ 2^{2^{n}} $\\
22.$ 2^{2^{n+1}} $\\

\item[3-4] Prove or disprove the following statements

a. $ f(n) = O(g(n)) $ implies $g(n) = O(f(n))$\\
False. Take $ f(n) = 1,g(n) = n $ and try the definition.

b. $ f(n) + g(n) = \Theta(\min(f(n),g(n)) $\\
False. Take $ f(n) = 1,g(n) = n $ and it fall back to the same relation in one, which can't be true.

c. $ f(n) = O(g(n)) $ implies that $ \lg(f(n)) = O(\lg(g(n))) $ if f(n) and lg(g(n)) are greater or equal to one for sufficiently large n.\\
True. From definition,\\
$ f(n) < c*g(n) \implies \lg(f(n)) < \lg c + \lg (g(n)) < \lg c * \lg(g(n)) $ for sufficiently large constant c.

d. $ f(n) = O(g(n)) $ implies that $ 2^{f(n)} = O(2^{g(n)}) $\\
False. Take $f(n) = 2n , g(n) = n $\\
There is no such constant to bound $4^n$ with $ 2^n $ as n goes large enough.

e. $ f(n) = O((f(n))^2) $\\
False. Let $ f(n) = 1/n $, then it is easy to see the contradiction since $ 1/f(n) $ doesn't converge.

f. $ f(n) = O(g(n)) $ implies $ g(n) = \Omega(f(n)) $\\
True. It can be proved by directly writing the definition.

g. $ f(n) = \Theta(f(n/2)) $\\
False. Let $f(n) = 2^n$. $ f(n)/f(n/2) = 2^{n/2} $, which doesn't converge.

h. $ f(n) + o(f(n)) = \Theta(f(n)) $\\
True. Set any positive constant c1 and add it by one. We can directly show that $ f(n) + o(f(n)) = O(f(n)) $\\
Set another positive constant c2 which is smaller than 1. We can directly show that $ f(n) + o(f(n)) = \Omega(f(n)) $\\
Thus this statement is true.

\item[3-5] a. Suppose that $f(n) = O(g(n)) $ and $f(n) = \mathop{\Omega}\limits^{\infty}(g(n)) $ are neither true.

Then for all positive constants c, there is $ f(n) > cg(n) $ for sufficiently large n. Arbitrarily select a positive constant c. There must be infinitely many ns such that $ f(n) > cg(n) $ and this contradicts the assumption that $f(n) = \mathop{\Omega}\limits^{\infty}(g(n)) $ is not true.

This property no longer hold if we replace $ \mathop{\Omega}\limits^{\infty} $ with $ \Omega $\\
Counter example: $ f(n) = n^{\sin(n)} $ and $ g(n) = \sqrt{n} $

b. I have the same feeling as the author of the CLRS study group did. I can't see anything meaningful to use the infinite Omega instead of the Omega relation since we usually analyze monotonic functions on n in real algorithm problems. The new notation might be helpful for dealing with oscillating functions. 

c. Suppose we don't change the definition of $ \Theta $ either. We can see that $ f(n) = \Theta(g(n)) \implies f(n) = O(g(n)) $ and $ f(n) = \Omega(g(n)) $\\
The converse way is also true because $ |f(n)|<cg(n) \implies f(n) < cg(n) $ which implies that $ f(n) = O(g(n)) $. Plus $ f(n) = \Omega(g(n)) $, the converse implication also holds.

d. We only need to put the lg term under f(n) and then change the direction of the inequality for soft Omega. The Theta definition is just the intersection of soft O and soft Omega

\item[3-6] Iterated functions\\
$$ f_c^*(n) = \min{i>=0: f^{(i)}(n)<=c} $$
\begin{tabular}{|c|c|c|}
\hline
f(n) & c & $ f_c^*(n) $\\
\hline
n-1 & 0 & $ \Theta(n) $\\
\hline
$\lg n$ & 1 & $ \Theta(\lg^*(n)) $\\
\hline
n/2 & 1 & $ \Theta(\lg n) $\\
\hline
n/2 & 2 & $ \Theta(\lg n +1) $\\
\hline
$\sqrt{n}$ & 2 & $ \Theta(\lg \lg n) $\\
\hline
$ \sqrt{n} $ & 1 & undefined behavior after $ n>1 $\\
\hline
$ n^{1/3} $ & 2 & $ \Theta(\log_3(\lg n)) $\\
\hline
$ n/\lg n $ & 2 & $ \omega(\lg\lg n), o(\lg n) $\\
\hline
\end{tabular}

I use the result of CLRS study group for the last line. I checked the correctness but I don't know whether there is a more tight bound for it.

Actually, I didn't get the point of dealing with some extremely intricate analysis, though the most examples looks helpful in applications.
\end{enumerate}
\end{document}